% !TEX root = MathFormula.tex
\chapter{Arithmetic}
\section{Set Theory}
\subsection{Basic Therms}
By putting objects with common properties into groups we create a {\em set}. 
Creation of sets is a basic function of our brain. The set of trees, set of animals and many many more.
The objects in such a set are calles {\em elements}.
\begin{example}\label{ex:set}
\[ \mathbb{A} = \{ 1, 2, 3, 5, 7, \} \]
\[ \mathbb{B} = \{ 2, 4, \dots , 2n \} \]
\[ \setQ = \{\frac{a}{b} \; | \; a,b \in \setZ \wedge b \ne 0 \} \] 
\end{example}

Elements of a set can also be points in $\mathbf{E}^n$ ($n$-dimensional space) and so such a set can be 
called {\em set of points}. In case the common property of such a set is a solution of a particular 
equation we are are already in a section about functions. Later more about this.

If every element $a_i$ of a set $\mathcal{A}$ is a member of a set $\mathcal{B}$, too, 
than we can say that $\mathcal{A}$ is an subset of $\mathcal{B}$.
\[ \mathcal{A} \subset \mathcal{B} \Leftrightarrow \mathcal{B} \supset \mathcal{A} \]

\begin{flalign}
\mbox{reflexive rule } & & \mathcal{A}\subseteq\mathcal{A},\;\; \emptyset\subset\mathcal{A} & \;\;\; \\
\mbox{transitive rule } & & \mathcal{A}\subset\mathcal{B}\wedge\mathcal{B}\subset\mathcal{C}\Rightarrow\mathcal{A}\subset\mathcal{C}\\
 & & \mathcal{A}\subseteq\mathcal{B}\wedge\mathcal{B}\subseteq\mathcal{A}\Leftrightarrow\mathcal{A}=\mathcal{B}\\
 & & \mathcal{A}=\mathcal{B}\;\forall\; x \in \mathcal{A} \wedge x \in \mathcal{B}\\
\mbox{symetrical rule } & & \mathcal{A}=\mathcal{B}\Leftrightarrow \mathcal{B}=\mathcal{A}\\
\mbox{transitive rule } & & \mathcal{A}=\mathcal{B}\wedge \mathcal{B}=\mathcal{C}\Rightarrow \mathcal{A}=\mathcal{C}
\end{flalign}

\begin{example}
\begin{itemize}
	\item The set $\{1, 2 \}$ is a proper subset of $\{ 1, 2, 3 \}$.
	\item Any set is a {\em subset} of itself, but not a {\em proper subset}.
	\item The empty set $\{ \} $, denoted by $\emptyset $, is also a subset of any given set $\mathbb{X}$. It is also always a proper subset of any set except itself.
	\item The set $\{x\; | \; x\in\setP\; \wedge\; x > 2 \}$ is a proper subset of 
	        $ \{ 2y+1 \; | \; y\in\setN \}$
\end{itemize}
\end{example}

\subsection{Set operations}

\begin{itemize}
\item[$\mathcal{A} \wedge \mathcal{B}$ ] the union of two sets contains all elements of $\mathcal{A}$ and $\mathcal{B}$. 
\item[$\mathcal{A} \vee \mathcal{B}$ ] the intersection contains elements which belong to $\mathcal{A}$ and $\mathcal{B}$. 
\item[$\mathcal{A} \setminus \mathcal{B}$ ] the relative complement of $\mathcal{B}$ in $\mathcal{A}$. contains all elements of of $\mathcal{A}$ what do not belong to  $\mathcal{B}$. 




\end{itemize}