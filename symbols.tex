\newcommand{\mySymbols}{\hspace*{0.2\textwidth}\=\hspace*{0.5\textwidth}\=\hspace*{0.2\textwidth}\kill}

\section{Symbols used in this book}
In fact \LaTeX\; is used to write this book. Often students and some other strange people have to write some kind of own papers and reports. For such particular event I did an overview for all of you with symbols used in this book and the corresponding commands in \LaTeX\; so can you easier put your own calculations on paper. But be aware: some commands are self made macros. All of such self made constructions are market in this listing with $\maltese$ and macros are printed in the listing on page \pageref{DSmyMacro} as a reference.
\subsection{Mathematical Symbols}
\subsubsection{Symbols of order}

\begin{tabbing}
\mySymbols
$ {1.}$               \> first element                   \> \verb` 1. `    \\
$ {, .}$              \> coma, point                     \> \verb` , . `   \\
$ {\dots}$            \> until...                        \> \verb` \dots ` \\
$ a_1,a_2,\dots ,a_n$ \> $a$ one, $a$ two, until $a$ $n$ \> \verb` a_1, `  \\
\end{tabbing}

\subsubsection{Equals and unequals}
\begin{tabbing}
\mySymbols
$ {\equiv}$           \> equivalent                        \> \verb` \equiv ` \\
$ {\ne}$              \> not equal                         \> \verb` \ne ` \\ 
$ {\sim}$             \> proportional                      \> \verb` \sim ` \\
$ {\approx}$          \> approximately                     \> \verb` \approx ` \\
$ {<}$                \> less then...                      \> \verb` < ` \\
$ {>}$                \> bigger then...                    \> \verb` > ` \\ 
$ {\leqslant}$        \> less or equal, not bigger then... \> \verb` \leqslant ` \\
$ {\geqslant}$        \> bigger or equal, not less then... \> \verb` \geqslant ` \\
$ {\ll}$              \> much less then                    \> \verb` \ll ` \\
$ {\gg}$              \> much greater then                 \> \verb` \gg ` \\
\end{tabbing}

\subsubsection{Elementary operations}
\begin{tabbing}
\mySymbols
$ +$                  \> addition                                \> \verb` + `       \\
$ -$                  \> subtraction                             \> \verb` - `       \\
$ \cdot \; \times$    \> multiplication                          \> \verb` \cdot  \times `      \\
$ \frac{\quad}{\quad}\; \div\;/ \; :$\> division                 \> \verb` \frac{}{} \div / : ` \\
$ \%$                 \> percent                                 \> \verb` \% `      \\
$ \permil$            \> permil                                  \> \verb` \permil ` \\
$ ()\; [ \;] \; \{ \} \; \langle \rangle$ \> all sorts of braces \> \verb` () [ ]  \{ \}  \langle \rangle ` \\
\end{tabbing}

\subsubsection{Symbols in Geometry}
\begin{tabbing}
\mySymbols
$ {\parallel}$                \> parallel                        \> \verb` \parallel `           \\
$ {\nparallel}$               \> not parallel                    \> \verb` \nparallel `          \\
$ {\uparrow\uparrow}$         \> parallel and unidirectional    \> \verb` \uparrow\uparrow    ` \\
$ {\uparrow\downarrow}$       \> parallel and bidirectional     \> \verb` \uparrow\downarrow `  \\
$ {\bot}$                     \> right angle                     \> \verb` \bot `                \\
$ {\triangle}$                \> triangle                        \> \verb` \triangle  `          \\
$ {\cong}$                    \> congruent                       \> \verb` \cong `               \\
$ {\sim}$                     \> similar                         \> \verb` \sim `                \\
$\measuredangle\;\sphericalangle$  \> angle                      \> \verb` \measuredangle \sphericalangle` \\
$ \overline{AB}$              \>                                 \> \verb` \overline{AB} `       \\
$ {\overset{\frown}{AB}}$     \>                                 \> \verb` \overset{\frown}{AB}` \\
\end{tabbing}

\subsubsection{Algebra and Elements of Analysis}
\begin{tabbing}
\mySymbols
 $ \sgn $                            \> signum              \>  $\maltese$\verb` \sgn ` \\
 $ |z|$                              \> ...                 \>  \verb` |z| ` \\
 $ \arc\,z$                          \> arcus $z$           \>  $\maltese$\verb` \arc ` \\
 $ n!$                               \> ...                 \>  \verb` n! ` \\
 $ \binom{n}{p}$                     \> ...                 \>  \verb` \binom{n}{p} ` \\
 $ \sum $                            \> ...                 \>  \verb` \sum  ` \\
 $ \prod $                           \> ...                 \>  \verb` \prod ` \\
 $ \sqrt{\quad}\;\sqrt[n]{\quad} $   \> ...                 \>  \verb` \sqrt[n]{x} ` \\
 $i$ or $j$                          \>                     \>  \verb` i j ` \\
 $\pi $                              \>                     \>  \verb` \pi ` \\
 $\e $                               \>                     \>  $\maltese$\verb` \e ` \\
 $()$                                \>                     \>  \verb` () ` \\
 $|\;|$ or $\det$                    \>                     \>  \verb` | | \det` \\
 $ $                          \> vector                    \> \verb` \vec{a} `\\
 $f(x)$                              \>                     \>  \verb` f(x) ` \\
\end{tabbing}

\subsubsection{Limits}
\begin{tabbing}
\mySymbols
 $\infty$                            \>                     \> \verb` \infty `  \\
 $(a,b)$                             \>                     \> \verb` (a,b) `  \\
 ${[a,b]}$ or $\langle a,b \rangle$  \>                     \> \verb` [a,b]or \langle a,b \rangle `  \\
 $\to $                              \>                     \> \verb` \to  `  \\
 $\lim $                             \>                     \> \verb` \lim `  \\
\end{tabbing}

\subsubsection{Differentials}
\begin{tabbing}
\mySymbols
 $\Delta f $            \>                     \> \verb` \Delta f  `  \\
 $f'(x)\, , f''(x) $    \>                     \> \verb` f'(x) f''(x) `  \\
 $f^{(n)}(x) $          \>                     \> \verb` f^{(n)}(x)  `  \\
 $\dot{\varphi}(t)\, ,\ddot{\varphi}(t)$
                        \>                     \> \verb` \dot{} \ddot{}`  \\
 $y',y'',y''' $         \>                     \> \verb` y' y'' y''' `  \\
 $ \d $                 \>                     \> $\maltese$ \verb` \d ` \\
 $\d f(x) $             \>                     \> \verb` \d f(x)  `  \\
 $\frac{\d y}{\d x},\frac{\d^2 x}{\d x^2},\dots \frac{\d^n y}{\d x^n} $ 
                        \>                     \> \verb` \frac{\d^n y}{\d x^n}  `  \\
 $f_x , f_y$            \>                     \> \verb` f_x  f_y  `  \\
 $\partial $            \>                     \> \verb` \partial  `  \\
 $f_{xx} , f_{xy} $     \>                     \> \verb`   `  \\
 $f_{yx} , f_{yy} $     \>                     \> \verb`   `  \\
 $\d f(x;y) $           \>                     \> \verb`   `  \\
\end{tabbing}

\subsubsection{Integral}
\begin{tabbing}
\mySymbols
 $\int $                \>                     \> \verb`   `  \\
 $\int f(x) \d x $      \>                     \> \verb`   `  \\
 $\int\limits^b_a f(x) \d x $
                        \>                     \> \verb`   `  \\
 $F(x)\big|^a_b $       \>                     \> \verb`   `  \\
 $\oint $               \>                     \> \verb`   `  \\
\end{tabbing}

\subsubsection{Exponential and logarithms functions}
\begin{tabbing}
\mySymbols
 $\exp x $             \>                     \> \verb`   `  \\
 $\log_a $             \>                     \> \verb`   `  \\
 $\lg $                \>                     \> \verb`   `  \\
 $\ln $                \>                     \> \verb`   `  \\
\end{tabbing}

\subsubsection{Trigonometric and Hyperboloid functions}
\begin{tabbing}
\mySymbols
 $\sin$                \>                     \> \verb`   `  \\
 $\cos $               \>                     \> \verb`   `  \\
 $\tan $               \>                     \> \verb`   `  \\
 $\cot $               \>                     \> \verb`   `  \\
\\
 $\arcsin $            \>                     \> \verb`   `  \\
 $\arccos $            \>                     \> \verb`   `  \\
 $\arctan $            \>                     \> \verb`   `  \\
 $\arccot $            \>                     \> \verb`   `  \\
\\ 
 $\sinh $              \>                     \> \verb`   `  \\
 $\cosh $              \>                     \> \verb`   `  \\
 $\tanh $              \>                     \> \verb`   `  \\
 $\coth $              \>                     \> \verb`   `  \\
\\
 $\arcsinh $           \>                     \> \verb`   `  \\
 $\arccosh $           \>                     \> \verb`   `  \\
 $\arctanh $           \>                     \> \verb`   `  \\
 $\arccoth $           \>                     \> \verb`   `  \\
\end{tabbing}

\subsection{Symbols of Sets}
\begin{tabbing}
\mySymbols
 $\mathcal{A}=\{a_k\}=\{a_1,a_2\dots a_k\}$ 
                       \>                     \> \verb`   `  \\
 $\in $                \>                     \> \verb` \in  `  \\
 $\not\in $            \>                     \> \verb` \not\in  `  \\
 $= $                  \>                     \> \verb` = `  \\
 $\{\,\},\,\emptyset$  \>                     \> \verb` \{ \}   \emptyset`  \\
 $\subset     $        \>                     \> \verb` \subset`  \\
 $\subseteq   $        \>                     \> \verb` \subseteq`  \\
 $\cup        $        \>                     \> \verb` \cup         `  \\
 $\cap        $        \>                     \> \verb` \cap         `  \\
 $\setminus   $        \>                     \> \verb` \setminus    `  \\
 $\rightarrow $        \>                     \> \verb` \rightarrow  `  \\
 $\times      $        \>                     \> \verb` \times       `  \\
 $\setP $              \>                     \> $\maltese$ \verb` \setP  `  \\
 $\setN $              \>                     \> $\maltese$ \verb` \setN  `  \\
 $\setZ $              \>                     \> $\maltese$ \verb` \setZ  `  \\
 $\setQ $              \>                     \> $\maltese$ \verb` \setQ  `  \\
 $\setR $              \>                     \> $\maltese$ \verb` \setR  `  \\
 $\setC $              \>                     \> $\maltese$ \verb` \setC  `  \\
 $\setH $              \>                     \> $\maltese$ \verb` \setH  `  \\
 $\setE $              \>                     \> $\maltese$ \verb` \setE  `  \\
\end{tabbing}

\subsection{Symbols of Logic}
\begin{tabbing}
\mySymbols
$ A_1 \Rightarrow A_2 $    \>  Implication ($A_1$ therefore $A_2$)     \> \verb` \Rightarrow `  \\
$ A_1 \Leftrightarrow A_2 $\>  Equivalence ($A_1$ equivalent to $A_2$) \> \verb` \Leftrightarrow `  \\
$ \neg A_1 $               \>  Negation (NOT)          \> \verb` \neg `  \\
$ A_1 \wedge A_2 $         \>  Conjunction (AND)       \> \verb` \wedge `  \\
$ A_1 \vee A_1 $           \>  Disjunction (OR)        \> \verb` \vee `  \\
$ A_1 \oplus A_2 $         \>  Antivalence (XOR)       \> \verb` \oplus `  \\
\end{tabbing}

\section{Other symbols}
\newcommand{\mySymbolss}{\hspace*{0.1\textwidth}\=\hspace*{0.4\textwidth}\=\hspace*{0.1\textwidth}\=\hspace*{0.3\textwidth}\kill}


If for any reason this symbols should be not enough for your paper there are plenty of other symbols available. I put this one I know here down and let you decide if it is for some use to you.
\subsection{Text Symbols}
\subsubsection{Special characters}
\begin{tabbing}
\mySymbolss
\dag   \>  \verb` \dag  ` \> \ddag   \>  \verb` \ddag `  \\
\S   \>  \verb` \S  ` \> \P   \>  \verb` \P `  \\
\i   \>  \verb` \i  ` \> \j  \>  \verb` \j `  \\
\copyright \> \verb` \copyrigh  ` \> \circledR  \>  \verb` \circledR `  \\
\textcircled{a} \> \verb` \textcircled{a} ` \> \texttrademark  \>  \verb` \texttrademark `  \\
\checkmark   \>  \verb` \checkmark  ` \> \maltese  \>  \verb` \maltese `  \\
\end{tabbing}
\subsubsection{Command characters}
\begin{tabbing}
\mySymbolss
\$   \>  \verb` \$  ` \> \&  \>  \verb` \$ `  \\
\#   \>  \verb` \#  ` \> \%  \>  \verb` \% `  \\
\{   \>  \verb` \{  ` \> \}  \>  \verb` \} `  \\
\_   \>  \verb` \_  ` \>   \\
\end{tabbing}
\subsubsection{Accents}
\newsavebox{\mySa}
\sbox{\mySa}{\makebox{\=a}}

\newsavebox{\mySb}
\sbox{\mySb}{\makebox{\'a}}

\newsavebox{\mySc}
\sbox{\mySc}{\makebox{\`a}}

\begin{tabbing}
\mySymbolss
\usebox{\mySb}     \>  \verb` \'a  `  \> \usebox{\mySc}    \>   \verb" \`a "    \\
\^a     \>  \verb` \^a  `  \> \"a    \>   \verb` \"a `    \\
\~a     \>  \verb` \~a  `  \> \usebox{\mySa}    \>   \verb` \=a `    \\
\u{a}   \>  \verb` \u{a} ` \> \v{a}   \>  \verb` \v{a} `  \\
\H{a}   \>  \verb` \H{a} ` \> \t{aa}  \>  \verb` \t{aa}`  \\
\r{a}   \>  \verb` \r{a} ` \> \b{a}   \>  \verb` \b{a} `  \\
\usebox{\mySd} \> \verb` \d{a} ` \footnotemark \> \c{a} \>  \verb` \c{a} `  \\
\end{tabbing}
\footnotetext{I admit: this is cheated. On page \pageref{DSmyMacro} you can see I did reuse the \mbox{\texttt{$\backslash$d\{a\}}} command for a math symbol. Here you can see what this command does in origin meaning. To preserve this command for this place it was necessary to save it in a \texttt{box} and let it drop dawn on this position.}


%\begin{tabbing}
%\mySymbols
% $ $                            \>                     \> \verb`   `  \\
%\end{tabbing}